
\cleardoublepage

\chapter{Conclusion}
\label{conclusion}

The results of this research show that the image-based mapping process is a viable option for locating and identifying individual plants.  For the experiment discussed in this paper, it is estimated that the image-based mapping allowed the field to be mapped four times faster <TODO> than traditional manual surveying, and with less tedious work.  The average error of 6 centimeters <TODO verify> is comparable to manual surveying errors, and the worse case error of 12 centimeters <TODO verify> is well within the maximum error allowed to differentiate adjacent plants.  The group identification method successfully located all but 2 QR codes, which is a success rate of 99.96\%.  While the mapping process was successful, other important conclusions regarding the platform, plant markers, cameras and system-complexities can be drawn from the research and are discussed below.

The image-based mapping is independent of the base platform is used, and the robot used in this experiment could easily be switched out for a simpler platform such as a manual push-cart.  A robot was used in this experiment based on the potential to automate the image collection step.  However, this research indicated two benefits of having a human present in the field during the image collection.  First is to detect if a camera stops taking pictures due to some error, and second is the ability to fix QR codes that are partially buried and covered by field debris.  The first issue could be mitigated by providing automatic feedback between the robot, user and data collection program and the second could be solved by walking through the field and fixing these codes before the robot drives itself through the field.  Both of these steps would be required for robust automated image collection.       

%7.3.2.	Using Tags instead of sticks.  Space out more.  Useful for plant visibility.  

TODO discuss cameras 
% Move faster
%7.3.3.	Faster cameras, lower latency
%7.3.1. Importance of paying close attention to camera settings, constant white balance.

% System complexity - ability to set the camera settings, setting thresholds, user interface for post-processing.  
% Other option - barcode scanner.  "trigger mapping". 

% Regardless of what type of platform used, important first steps in creating a mapping process that will allow... 
The success of this new mapping process will allow researchers to increase the size of their experiments which will allow more variety in plant populations and an increase in the number of field repetitions among certain plant groups. This will lead to better selection and faster refinement of  

