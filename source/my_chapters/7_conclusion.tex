
\cleardoublepage

\chapter{Conclusion}
\label{chapter:conclusion}

The results of this research show that the image-based mapping system is a viable option for locating and identifying individual plants.  The average absolute Euclidean error of 5.1 centimeters is comparable to manual surveying errors, and the worst measured longitudinal error of 10.9 centimeters is well within the maximum error allowed to differentiate adjacent plants.  The group identification method successfully located all but 2 QR codes, which is a success rate of 99.96\%.  While the mapping process was successful, other important conclusions regarding the platform, plant markers, cameras, and system-complexities can be drawn from the research and are discussed below.

The image-based mapping is mostly independent of the base platform that is used, and the robot could easily be switched out for a simpler platform such as a manual push-cart.  A robot was used in this experiment based on the potential to automate the image collection step.  This research, however, indicated two benefits of having a human present in the field during the image collection.  First is to detect if a camera stops taking pictures due to an internal error, which did occur several times during the experiment, and second is the ability to fix QR codes that are partially buried and covered by field debris.  The first issue could be mitigated by providing automatic feedback between the robot, user, and data collection program, and the second could be solved by walking through the field and fixing these codes before the robot makes its passes.  Both of these steps would be required for robust automated image collection.       

The least successful aspect of the experiment was the use of dyed blue sticks acting as plant-markers.  These wooden sticks suffered from inconsistent intensity and didn't reflect enough light to appear saturated in the images.  An alternative option is to use colored, plastic square markers that are pierced through un-dyed wooden sticks.  To avoid large amounts of plastic being distributed in the field these markers would not need to be on every plant, but rather every 4 or 5 plants.  If there is little field debris or weeds then these plant markers may not be necessary at all. 

The limiting factor on how quickly images could be collected was the maximum trigger rate of the Canon 7D cameras.  Additionally, these cameras had an indeterminate latency that led to positional errors and suffered from occasional triggering errors.  A potentially better option would be to use a camera originally designed to transfer images over a \acf{usb} port and equipped with a high-resolution lens.  These cameras typically have much higher capture rates and were intended to be used for scientific or robotic applications, unlike the Canon cameras which were primarily designed for photographers. 

Perhaps the biggest conclusion drawn from the research is the importance of managing complexity in an application that is primarily intended to be used by researchers.  Using images adds technical complexity in the fact that camera settings have to be setup carefully, and the thresholds in the post-processing have to be chosen to work the color spectrum of the images.  These thresholds can vary over multiple experiments due to slight changes in environmental lighting, and it's difficult to find a set of values that work well for all scenarios.     

Regardless of what type of equipment is selected to collect images, this thesis presents important first steps in creating a robust mapping system for not only mapping transplant coordinates but also assigning them to different groups. This functionality will potentially allow field sizes to scale up while also leveraging automated technologies needed to meet the ever growing demand for food.
