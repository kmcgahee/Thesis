
\cleardoublepage

\chapter{System Design}
\label{chapter:system_design}

A large part of the overall mapping system consists of the various components needed to capture, organize, and locate images of the field, as well as any additional field items needed for identifying plants.  The proper design of this part of the mapping system is the perhaps the most important step in the mapping process.  Poor design leads to insufficient image quality, missing field coverage, or improperly geo-referenced images, which no amount of post-processing can correct.  

The first part of the system discussed is the markers used for plant identification because the required size of these markers impose constraints on the rest of the system.  Next, the base platform and additional equipment, such as the cameras, are presented along with reasoning about nominal parameters such as vehicle speed.  This chapter concludes by describing additional field markers that are not strictly necessary but help improve the robustness of the mapping system. 

\section{Plant Identification}
\label{section:plantid}

As mentioned in the introduction chapter, the mapping process involves not only determining plant coordinates but also assigning each plant to a group.  Each plant group is referenced by a unique identifier (ID), such as 1035.  The method used in this research is to encode the ID in a two-dimensional barcode that is placed at the beginning of each group of plants in the field. 

If a plant group needs to be planted in different parts of the field then a repetition letter is appended to the group number.  For example, three codes containing the text 1035A, 1035B, and 1035C all belong to plant group 1035.  Since the two-dimensional barcodes are only placed at the beginning of the group, it's critical to know the direction of planting for each row in order to associate the correct set of plants with the correct ID.  Codes could potentially be placed on both sides of each group to remove this added challenge, but this doubles the amount of code construction time and chance of missing a code during the post-processing.  Instead, the row direction is encoded in row markers which are discussed later in this chapter.

\subsection{Code Format}
\label{section:code_format}

\acf{qr} codes were selected as the barcode format. This is a standardized format that was made publicly available by Denso Corporation over 20 years ago.  It was first used for item tracking in the Japanese automotive industry and has most recently become well known for encoding \acp{url} for websites ~\citep{Denso:2014}. Two important characteristics of this format are that it can be read from any orientation, and it can still be read if part of the code is damaged.  Various other types of formats, such as Aztec or Micro \ac{qr}, can encode information in smaller grids by restricting the character encoding, but offer less error correction.  Also, at the time this research was conducted these alternate formats were not supported by any of the open-source readers that were investigated. 

\begin{figure}
	\centering
    \includegraphics[width=3in]{figures/generated_codes_1035.jpg}
    \caption[2D barcode formats]{Comparison of 2D barcode formats encoding the text 1035 with high error correction.  From left to right: \acf{qr}, Aztec, and Micro \ac{qr}.}
    \label{barcode_formats}
\end{figure} 

\subsection{Size Constraint}

For fields with thousands of different plant groups it's not feasible to place these \ac{qr} codes by hand.  Therefore, they must fit through the deposit cylinders of the transplanter shown in Figure~\ref{figure:transplanter}.  In order to not get caught in the cylinders, the codes cannot be larger than 2.5 centimeters in diameter. Since there needs to be a white margin around the actual \ac{qr} code, the code grid ends up being roughly 2 centimeters on each side.  The \ac{qr} format is split into different versions that define how many squares make up the code.  The first, and smallest, version is a grid of 21 by 21 squares.  An example of this first version is shown in Figure~\ref{barcode_formats}.  This sized grid results in a maximum square size of only 1 millimeter.  

\subsection{Code Construction}

The codes must be easy to produce due to the potentially large number of codes required for each field.  The NiceLabel program can be used to generate all the \ac{qr} codes at once, and a thermal printer, such as the Stover model SM4.25, can rapidly print the codes on pot labels.  However, the pot labels need a solid base to stay upright and grounded during transplanting. The researchers at the Land Institute developed a degradable cement base into which the pot labels are inserted.  This base consists of 2 parts perlite, 1 part water, and 1 part cement.  An example of one of these pot label \ac{qr} codes can be seen in Figure~\ref{QR_code}.  While the plastic pot labels are not degradable, it is in theory possible to re-use them between experiments if they are gathered up after the mapping is complete.

\begin{figure}
	\centering
    \includegraphics[height=2in]{figures/qr_code_407.jpg}
    \caption[Pot label QR code]{\ac{qr} code printed on pot label inserted in cement base.}
    \label{QR_code}
\end{figure}

\section{Platform Design}
\label{section:platform_design}

There are many types of platforms that could be used for mapping.  Aerial vehicles have the benefit of autonomously traversing the field without the challenge of avoiding plants. However, they are unable to provide external lighting or shading which is important when post-processing the images. In addition, the size constraint on \ac{qr} codes would require low altitude flights at a constant altitude to keep the codes properly focused, which is not easy to achieve.  Higher altitude flights would be possible with a telescopic lens, but this increases cost, weight, and most critically the effects of error in camera orientation.  For these reasons, only ground platforms are considered.

Two different types of ground platforms are investigated, a manual push-cart and a four-wheel robotic vehicle.  The push-cart excels in its simplicity, however for this thesis only the robotic platform is discussed.  The main benefits of a robot is the ability to drive at a constant speed and the option to operate autonomously.  Driving at a consistent speed is important to ensure sufficient overlap between successive images, and a self-driving vehicle removes much of the tedious work associated with imaging large fields.  

\subsection{Robotic Platform}

The selected robotic platform is the Husky A200 mobile robot made by Clearpath Robotics.  The Husky is a four-wheeled, differential drive robot measuring 39 inches in length and 27 inches wide.  It features a maximum speed of 1 meter per second, and it can carry up to 75 kilograms in optimal conditions.  A custom C-channel structure was added to the top of the robot to enable it to image the field.  This structure can be seen in Figure~\ref{husky_rocky_ford}.  Attached to the front of the structure are two Canon 7D \ac{dslr} cameras.  Using two cameras allows two rows to be mapped at the same time. 

\begin{figure}
	\centering
    \includegraphics[height=2.7in]{figures/sunflower_rocky_ford_labeled_small.jpg}
    \caption[Husky robot]{Husky mobile robot equipped with two cameras.}
    \label{husky_rocky_ford}
\end{figure}

On the back of the C-structure are two white Trimble AG25 antennas that attach to a Trimble BX982 \ac{gnss} receiver.  This receiver is mounted to the top of the robot.

\subsection{GNSS Receiver}

The BX982 receiver provides centimeter level accuracy when paired with a fixed base receiver broadcasting \ac{rtk} correction signals.  For this application the fixed base is a Trimble Ag542 receiver with a Trimble Zypher Geodetic antenna, which is shown in Figure~\ref{base_station}.  This base station communicates with the SNB900 rover radio mounted on the robot over a 900 megahertz radio link.   The dual antenna design allows the robot to determine its heading to within approximately 0.1 degrees.  This accurate heading is important for geo-locating plants and \ac{qr} codes within images, as well as allowing the robot to operate autonomously. 

\begin{figure}
	\centering
    \includegraphics[height=2in]{figures/sunflower_base_cropped_small.jpg}
    \caption[Base station with tripod]{Ag542 base station mounted on tripod.}
    \label{base_station}
\end{figure}

\subsection{Cameras and Lighting}

The Canon 7D features an 18 megapixel sensor and is fitted with a fixed 20 millimeter focal length, wide-angle lens.  The camera contains an \ac{apsc} sensor rather than a full frame 35 millimeter sensor. When paired with the wide-angle lens this gives a horizontal angle of view of 58.3 degrees and a vertical view of 40.9 degrees.
  
As shown in Figure~\ref{husky_rocky_ford}, the cameras are mounted far out in front of the robot.  This is so the wheels and front bumper do not show up in the image and effectively reduce the field of view.  In addition, each camera is mounted with a 90 degree yaw offset with respect to the platform so that the longer side of the image is aligned with the forward movement of the robot as seen in Figure~\ref{figure:image_fov}.  

\begin{figure}
	\centering
    \includegraphics[height=2in]{figures/camera_directions.jpg}
    \caption[Camera field of view]{Depiction of camera field of views relative to robotic platform.}
    \label{figure:image_fov}
\end{figure}

An item that is not pictured on the robot, but is shown in Figure~\ref{figure:canon_and_bars}, is a \ac{led} bar.  These 9 watt bars provide external lighting at night for consistent scene illumination.  It is feasible to only use 2 bars, one for each camera, however using 4 bars provides enough light for the camera settings to be set conservatively which improves the robustness of the post-processing pipeline.  Another benefit of using one bar on each side of each camera is it noticeably reduces image glare on the \ac{qr} codes.  

\begin{figure}
	\centering
    \includegraphics[height=2in]{figures/canon7d_and_LEDs.jpg}
    \caption[Canon 7D and LED bars]{Canon 7D and external lighting bars.}
    \label{figure:canon_and_bars}
\end{figure}

\subsection{Determining Parameters}
\label{section:determining_parameters}

There are a number of parameters that need to be carefully chosen to ensure quality images and sufficient field coverage.  The most important to determine first is the camera height, as that determines the resolution of the image.  If the image resolution is too low then the \ac{qr} codes will be unreadable.  

If each pixel could correspond to exactly one square on the \ac{qr} code then the minimum resolution would be 1 pixel per millimeter, since each square is roughly 1 millimeter in size.  In reality, this is almost never the case, and the theoretical minimum is 3 pixels for one square.  If only 2 pixels are used then the light from a black square could be split in half with the adjacent white square, and the result would be an unresolved gray square.  However, all lenses also introduce some loss in contrast.  The amount of contrast lost is a function of many things such as lens diffraction, aperture, and pixel position.  Also, the camera's optical axis is not always perfectly orthogonal to the code.  To account for these extra effects, the actual minimum resolution is set to 5 pixels per millimeter.  

The 18 million pixels on the sensor are split into a 5184 by 3456 grid.  The minimum resolution then requires a maximum image size of 1037 by 690 millimeters, which constrains the cameras to be no higher than 930 millimeters above the \ac{qr} codes.  In order to make the imaging process more robust, the cameras are mounted 700 millimeters above the ground which corresponds to an image size of 781 by 520 millimeters (or 31 by 21 inches), and a resolution of approximately 6.5 pixels per millimeter.  Mounting the cameras lower than the maximum requirement also allows the external lighting to be more concentrated and requires smaller shading if the imaging is done during the day.  

Another requirement that is added to make the mapping process more robust is each \ac{qr} code must be in a minimum of two images.  This helps solve temporary issues such as insects flying in front of the camera as well as offers multiple perspectives if the code is planted at an angle.  In order to ensure this, the maximum spacing between successive images is given by the equation
\begin{align*}
 \text{max spacing} &= (\text{image width} - \text{QR side length} - \text{pad}) / 2 \\
             &= (781 - 25 - 75) / 2 \\ 
             &= 340 \text{ millimeters}
\end{align*}
where the 'pad' is extra spacing to account for variations in camera latency.
  
An additional constraint on the cameras that must be taken into account is the minimum trigger period.  Many cameras, including the Canon 7D, are capable of exposing images rapidly and then buffering them before they are processed.  However, the minimum trigger period considered in this section is the minimum amount of time for an image to be exposed, processed, and saved without continued buffering, as buffering can only be sustained for short periods.  This was experimentally determined for the Canon 7D to be 0.7 seconds.  

In order to satisfy the maximum image spacing of 340 millimeters, while not exceeding the minimum trigger period, the robot cannot drive faster than 0.5 meters per second.  One downside of driving this fast is cameras begin to noticeably shake when the field is not smooth, which can lead to blurry images.  Therefore the nominal robot speed is set to 0.4 meters per second.  These platform parameters are summarized in Table~\ref{table:platform_params}.

\begin{table}
    \begin{center}
    \caption{Summary of platform parameters.}
    \begin{tabular}[c]{|c|c|c|}
        \hline
        Parameter & Value & Units \\
        \hline
        Camera Height    & 700       & millimeters         \\
        Image Size       & 781 x 520 & millimeters         \\
        Image Resolution & 6.5       & pixels / millimeter \\
        Trigger Period   & 0.7       & seconds / image     \\
        Robot Speed      & 0.4       & meters / second     \\
        \hline
    \end{tabular}
    \label{table:platform_params}
   \end{center}
\end{table}

\subsection{On-board Computers}

There are a total of four computers used on the Husky.

\begin{description}
\item[Main Husky Computer] - custom mini-ITX situated inside the main compartment of the robot running Ubuntu 12.04 along with the \ac{ros}. This computer contains all of the telemetry and guidance logic and is discussed more in Section~\ref{section:base_functionality}.  
\item[Husky Microcontroller] - Atmel ARM-based SAM7XC256 enclosed in the back of the robot chassis.  This microcontroller receives linear and angular velocity commands from the main computer and reports feedback from the robot's encoders, battery, and motors. 
\item[Husky Interface Computer] - Lenovo T400 laptop that is also running Ubuntu 12.04 along with \ac{ros}.  This computer sits on top of the Husky and is used to send commands to the robot using a \ac{ssh}.
\item[Data Collection Computer] - Lenevo S431 laptop running Windows 7 that also sits on top of the Husky.  This computer runs the program responsible for saving data from the cameras and \ac{gnss} receiver.   
\end{description}

For simplicity, the output of the \ac{gnss} receiver is split to both the data collection computer as well as the main Husky computer.  It would be ideal to run the data collection program on the Husky interface computer to eliminate the need for an extra computer.  However, the data collection program requires a Windows based operating system to interact with the Canon cameras, and at the time of this research \ac{ros} was not stable on Windows. 

\section{Guidance and Control}

The Husky arrived from Clearpath with basic driving functionality, but this was not sufficient for the mapping system.  This section describes the additional functionality developed for the robot that enables it to operate in autonomous or semi-autonomous modes.  

\subsection{Base Functionality}
\label{section:base_functionality}

When the Husky was purchased the main computer came installed with the \acl{ros}, which is popular open-source framework that provides many of the same services as a traditional operating system, such as inter-process communication and hardware-abstraction.  One major benefit is \ac{ros} allows different functionality to be split up into separate processes, referred to as nodes.  This promotes code re-use and prevents one component from crashing the entire system.   The nodes that were pre-installed on the Husky are listed below.

\begin{description}
\item[Teleop node] - receives driving commands from the Logitech controller shown in \ref{figure:cruise_control}.  The default functionality is when the X button is held down then the left and right analog sticks command linear and angular velocity, respectively.
\item[Husky node] - in charge of sending the velocity commands over a serial port to the microcontroller that controls the motors.  
\item[IMU node] - driver for the UM6 orientation sensor that came installed on the robot.  This node is not used since the platform is stable, and the multiple \ac{gnss} antennas determine yaw.
\end{description}

\subsection{Cruise Control}

When driving through the field it's important that the robot maintains a constant speed to ensure all \ac{qr} codes and plants are imaged.  With the basic teleop functionality this was difficult to achieve while also keeping the robot centered in the middle of the row. To solve this issue, the teleop node was extended to include cruise control functionality that is commonly seen in consumer automobiles.  

\begin{figure}
	\centering
    \includegraphics[width=5in]{figures/logitech_controller_labelled.jpg}
    \caption[Cruise control buttons]{Logitech controller for Husky showing cruise control functionality.}
    \label{figure:cruise_control}
\end{figure}

Cruise control can be enabled by either pressing both the Enable 1 and Enable 2 buttons at the same time or by the Preset button.  The Preset button defaults to a certain configured speed, by default 0.4 meters per second.  If the Override button is pressed then the linear speed is temporarily determined from the left analog stick, as is the case in the basic driving mode.  The Resume button returns to the last speed, if there was one, and the up and down arrows on the D-pad vary the commanded speed in increments of 0.05 meters per second.    

\subsection{Automated Control}
\label{section:automated_control}

While this cruise control feature makes it feasible to manually drive the robot through the field, for large experiments this is a tedious task that requires ten or more hours of keep the robot centered between the rows.

\ac{ros} contains well-developed navigation functionality that allows the robot to convert odometry and sensor data into velocity commands.  This navigation code, unfortunately, is primarily based around advanced functionality such as cost maps, detailed path planning, and map-based localization. All of which are unnecessary for this application.  Therefore, the author decided to develop a simple guidance solution that is implemented in the following nodes:

\begin{description}
\item[GPS node] - combines position and yaw data from the \ac{gnss} receiver and publishes this data to the rest of the \ac{ros} system.
\item[Waypoint Upload node] - allows the Husky interface computer to load a set of waypoints into to robot.
\item[Guidance node] - computes robot velocity needed to follow a set of waypoints. 
\end{description}

\section{Data Collection Software}
\label{system-software}

A critical part of the mapping process is being able to accurately associate each image with the position and orientation of the camera at the time the image was taken.  This process was accomplished using the Dynamic Sensing Program (DySense) which is an open-source data collection program that provides the means to obtain, organize, and geo-locate sensor data.  This program was developed by the author in order to standardize data collection across various types of platforms and sensors.  A screen shot of the DySense user interface can be seen in Figure~\ref{dysense_screenshot}.

Similar to how \ac{ros} splits different functionality into processes, DySense can split sensor drivers into processes which allows them to be written in any programming language.  The camera sensor driver is implemented in $C\#$ and uses the \ac{edsdk} to interact with the camera.  This driver allows the images to be downloaded from the camera in real-time and assigns each one a unique file name and an estimated \ac{utc} stamp of when the image was exposed.

\begin{figure}
	\centering
    \includegraphics[width=6in]{figures/dysense2.jpg}
    \caption[Data collection program]{Screenshot of the data collection program.}
    \label{dysense_screenshot}
\end{figure}

\section{Additional Markers}
\label{system-markers}

In addition to the plant group \ac{qr} codes, there are two other types of markers used in the mapping process. Row markers and plant markers.   

\subsection{Row Markers}

Row markers are placed at the beginning and end of each row. Similar to identifying plant groups, these markers are also implemented using \ac{qr} codes.  The row codes store the row number and signify whether the code is the start ("St") or end ("En") of the row. As discussed in Section~\ref{section:plantid}, it is critical to know the planting direction of each row because that defines which plants are associated with each group ID.

\begin{figure}
	\centering
    \includegraphics[height=2in]{figures/row_codes.jpg}
    \caption[Row QR codes]{\ac{qr} codes marking the start and end of row 21.}
    \label{figure:row_codes}
\end{figure}

\subsection{Plant Markers}
\label{section:plant_markers}

The second type of marker is used to help distinguish plants from other debris that may be found in the field after tilling.  This additional marker is optional, it but helps improve the robustness of locating plants.  Depending on the size and quality the field these markers can be used on every plant or, for example, every 4 plants. 

One type of plant marker is a blue dyed wooden stick approximately 5 inches in length which is placed in the center of each plant. The color blue is selected because it provides the largest difference between other hues likely found in the field, such as yellow/green in plants and red in soil.  In addition to marking the plants, this stick also helps prevent the plants from flipping over when exiting the planter.

Another type of plant marker worth mentioning is a colored tag pierced through the top of an un-dyed wooden stick.  This type of marker can provide addition information for manual plant inspection and is much more saturated than the dyed sticks, making it easier to detect in the post-processing pipelines. 

\begin{figure}
	\centering
    \includegraphics[height=2.5in]{figures/plant_markers.jpg}
    \caption[Plant markers]{Examples of blue stick (left) and colored tag (right)}
    \label{figure:plant_markers}
\end{figure}
