
\cleardoublepage

\chapter{System Design}
\label{system}

TODO

\section{Plant Identification}
\label{system-plantid}

A common way to encode information in a machine-readable format is to use a two dimensional barcode.  These barcodes 

what we're using it for

\subsection{Code Format}

For this project Quick Response (QR) codes were selected as the barcode format. This is a standardized format that is license free and allows for error correction if part of the code is unreadable.  What I'm encoding

Figure 

\subsection{Size Constraints}

Need to make small...  Mention physical size to fit through transplanter.
The standard defines 40 different sizes of codes ranging from a 21x21 grid up to a 170x170 grid.  The size, or version, is automatically chosen 

Various other types of formats, such as <> or <>, allow information to be encoded in slightly less number of squares, but at the time of the research these formats weren't supported by any of the open-source readers that the researcher investigated.  

\section{Platform Design}
\label{system-platform}

The base platform selected is the Husky A2000 mobile robot made by Clearpath Robotics.  The Husky is a four wheeled differential drive robot measuring 30 inches long and 25 inches wide.  As seen in figure TODO a custom C-channel structure was added onto the top of the robot to enable it to image the field.  Attached to the front of the top structure are two Canon 7D digital single-lens reflex (DSLR) cameras. On the back are two antennas <TODO what kind> which attach to a Trimble BX982 global navigation satellite system (GNSS) receiver which is mounted to the top of the robot.   

\subsection{GNSS Receiver}

The BX982 receiver provides centimeter level accuracy when paired with a fixed base receiver broadcasting RTK correction signals.  For this application the fixed base is a Trimble Ag542 receiver with a <> antenna as shown in figure TODO.  The robot receiver and base receiver communicate over a <> MHz radio link.   The dual antenna design allows the robot to determine its heading, or yaw, to within approximately 0.1 degree.  This accurate yaw is important for geo-locating plants and QR codes within images, as well as allowing the robot to operate autonomously as discussed in section <TODO>. 

\subsection{Camera and Lighting}

The Canon 7D contains a <> megapixel sensor and is fitted with a fixed 20mm focal length wide-angle lens.  The Canon 7D contains an Advanced Photo System type-C (APS-C) sensor rather than a full frame 35mm sensor, which paired with the 20mm lens gives a horizontal angle of view of <> degrees and a vertical view of <> degrees.  

Camera placement

Talk about LED lights.  

\subsection{Nominal Speed}

Overlap
For redundancy and improved accuracy

\subsection{On-board Computers}

There are a total of four computers used on the platform.  The first is located within the main compartment of the Husky a custom mini-ITX computer running the Robot Operating System (ROS) which is discussed more in section TODO.  
Mention wifi/router/laptops.  

\section{Guidance and Control}
\label{system-modes}

These packages are found 

\section{Base Functionality}

When the Husky was purchased the main computer came installed with the Robot Operating System (ROS), which is popular open-source framework that provides many of the same services as traditional operating system such as inter-process communication and hardware-abstraction.  One major benefit of ROS is it allows different functionality to be split up into separate processes, referred to as Nodes.  This promotes code re-use and prevents one component from crashing the entire system.   The Nodes that were pre-installed on the Husky are listed below

IMU Node
Teleop Node
Husky Node

These base packages allowed the robot to be manually driven by the logitech controller shown in figure TODO.  The buttons did <> and the joystick sent <> commands.  

\section{Cruise Control}

However, when driving through the field it's important to maintain a constant speed to ensure all QR codes and plants are imaged.  With the basic teleop functionality this was difficult to achieve while also keeping the robot centered in the middle of the row. To solve this issue the researcher extended the teleop node to include cruise control functionality that is commonly seen in automobiles.  TODO explain the buttons.  

\section{Automated Control}

While this cruise control feature made it feasible to manually drive the robot through the field, for large experiments this was a tedious task that required ten or more hours of keep the robot centered between the rows.  

path planning
localization
motion

ROS contains well-developed navigation functionality that allows the robot to convert odometry and sensor data into velocity commands for the robot.  However this navigation is based around advanced functionality such as cost maps, detailed path planning and map-based localization, all of which are unnecessary for this application.  Therefore the researcher decided to implement a simple, custom guidance solution that is implemented in the following nodes

GPS
Waypoint Upload
Guidance

Required <waypoints>
For simplicity no filtering was used . For faster speeds it may be necessary.

\section{Data Collection Software}
\label{system-software}

A critical part of the mapping process is being able to accurately associate each image with the position and orientation of the camera at the time the image was taken.  

This process was accomplished using the Dynamic Sensing Program (DySense), which is an open-source data collection program that provides the means to obtain, organize and geo-locate sensor data.  This program was developed by the researcher in order to standardize data collection across various types of platforms and sensors.  A screen shot of DySense can be seen in figure <>.

Similar to how ROS splits up different functionality into processes, DySense can split sensor drivers into processes which allows them to be written in any popular programming language.  The camera sensor driver was written in C# and used the EOS Digital Software Development Kit (EDSDK) to interact with the camera.  This driver allows the images to be downloaded from the camera in real-time, given a unique file name and assigned a time stamp of when the image was exposed.

TODO go over how data is collected

storing height above ground and height above ellipsoid separately.   

TODO include figure of DySense

\section{Additional Markers}
\label{system-markers}

In addition to the group QR codes there are two other types of markers used in the mapping process.  The first is a row end marker which is also represented as a QR code.  These row codes store the row number and signify whether the code is the start or the end of each row.  It's important to know the planting direction of each row because that defines which plants are associated with each group QR code.   

\subsection{Row Codes}

row code size.

\subsection{Plant Markers}

The second type of marker is used to mark plants to help distinguish them from other plant debris that may be found in the field after TODO.  The marker used in this experiment was a blue dyed wooden stick approximately 5 inches in length that was placed in the center of each plant. The color blue was selected because it provides the largest difference between other hues likely found in the field, such as yellow/green in plants and red in soil.  In addition to marking the plants, this stick also helped prevent the plants from flipping over when exiting the planter.

TODO include figure of blue stick and row code
