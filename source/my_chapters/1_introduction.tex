
\cleardoublepage

\chapter{Introduction}
\label{introduction}

Recent research shows the growth rate of crop production must increase in order to meet the demand created by a quickly growing world population (todo ref source).  This increase in production is supported by new, automated technologies that allow both researchers and farmers to operate more efficiently.  One such technology, which is investigated in this research, is the means to automatically develop a map of plant locations within a field.  This field map enables other useful technologies, such as precise weed spraying or easily referencing plants in a database. 
 
An important point to consider with field mapping is there are two different starting points.  In some applications seeds are directly planted in the field, while in other applications plants are grown, typically in a greenhouse, and then planted into the field.  The latter process is known as transplanting, and the plants are referred to as transplants.  In addition, certain applications require similar plants to be identified as part of a larger plant group.  These similarities could be based on crop type or genetic variations. 

The task of mapping transplants is traditionally performed by manually walking through the field and surveying the location of each plant. However, this is a tedious process that does not scale well to large fields containing tens of thousands of plants.  Several automated mapping processes, such as ~\citep{Perez-Ruiz:2012} and ~\citep{Soille:2000}, have been explored, but each of these methods have key drawbacks which are discussed in section \ref{section:similar_research}.  The research presented in this thesis aims to address these shortcomings by developing a novel mapping system based on computer vision techniques.  This image-based mapping system, especially when used with an automated robotic platform, is an effective solution for locating and identifying transplants. 

\section{Mapping and Identification}

The mapping process, which refers to creating a field map, is split into two smaller processes.  The first of these is to determine the coordinates of each plant in the field with respect to one or more reference frames.  Depending on the application these could be a local field reference frame or a global frame, such as one specified by latitude and longitude.  Since the plants are fixed to the earth this is a two-dimensional mapping problem and a third dimension, such as altitude, is not required.  In addition to assigning coordinates, it's also useful to determine the row number that each plant is found in.  This first sub-process is referred to as plant mapping. 

Since plants in the field may not all be identical, for example they may have variations in their genetics, then this requires the mapping process be able to assign a unique group label to each plant.  These plant groups, or plant families, are known prior to planting and the challenge is to determine where each plant group ends up in the field.  This process of assigning a plant to a group of plants is referred to as plant identification. The ability to handle this additional requirement is one the key improvements of the proposed mapping solution over previous approaches.

\section{Applications}

There are many applications where a field map is useful.  One application is for perennial agriculture where plants die and then re-grow each year.  Having a field map with accurate coordinates allow the same plants to be tracked over multiple years.

An extension of this application, which applies to both perennial and annual agriculture is using the plant coordinates to automate maintenance tasks.  For example, herbicides or pesticides can be applied between more efficiently. ~\citep{Carballido:2013} As well as other common tasks, such as intra-row tilling or cutting, can be made more effective by automatically avoiding plants. ~\citep{Bakker:2010}  

Another application that applies to both types of agriculture is automatic plant lookup within a database.  For example, if a researcher needs to record notes about a plant in an un-mapped field they would likely need to manually enter, or possibly scan, a plant number to retrieve that plant's entry within the database.  However, if they are equipped with a \ac{rtk} receiver they would be able to automatically retrieve the plant's entry based on their current location in the field. 

Lastly, a field map can be combined with geo-tagged sensor measurements, such as soil moisture content or height readings, to associate the measurements with individual plants.  This automated data collection greatly increases the amount of plant traits that can be consistently measured. ~\citep{Ruckelshausen:2009}

\section{Mapping System Overview} 

The image-based mapping system can be broken down into three parts.  The first is the mapping equipment, which includes the platform that traverses the field as well as the cameras used to collect images.  The second part consists of additional items placed in the field that enable plant identification or make the overall mapping process more robust.  The combination of these first two parts is referred to as the system design, which is described in chapter \ref{chapter:system_design}. 

The third part of the mapping process is converting the images into a useful field map.  This involves using image-processing techniques to extract meaning information from the images, such as the locations of plants, as well as other grouping and clustering algorithms.  These algorithms are applied in sequential steps, which collectively are referred to as the post-processing pipeline.  This section of the mapping system is covered in chapter \ref{chapter:pipeline}. 

The next two chapters of this thesis, \ref{chapter:experiment} and \ref{chapter:results}, discuss applying the mapping process to a large-scale experiment conducted at the Land Institute in Salina, Kansas.  The final chapter summarizes the findings of the research and covers the important lessons learned during this experiment.
