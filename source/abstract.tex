% +--------------------------------------------------------------------+
% | Abstract Page
% +--------------------------------------------------------------------+

\pagestyle{empty}
%\vspace{1cm}
\setlength{\baselineskip}{0.8cm}

%\indent

% +--------------------------------------------------------------------+
% | Enter the text of your abstract below, maximum of 500 words.
% +--------------------------------------------------------------------+


Developments in farm related technology have increased the importance of mapping individual plants in the field.  An automated mapping system allows the size of these fields to scale up without being hindered by time-intensive, manual surveying.  This research focuses on the development of a mapping system, which uses geo-located images of the field to automatically locate plants and determine their coordinates.  Additionally, this mapping process is capable of differentiating between groupings of plants by using \ac{qr} codes.  This research applies to green plants that have been grown into seedlings before being planted, known as transplants, and for fields that are planted in straight rows. 

The development of this mapping system is presented in two stages.  First is the design of a platform equipped with a \ac{rtk} receiver that is capable of traversing the field and capturing images. Second is the post-processing pipeline, which converts the images into a field map.  This mapping system was applied to a field at the Land Institute containing roughly 25,000 transplants. The results show the mapped plant locations are accurate to within a few inches, and the use of \ac{qr} codes is effective for identifying plant groups.  These results demonstrate this system is successful in mapping large fields.  However, the high overall complexity makes the system restrictive for smaller fields, where a simpler, semi-automated solution may be preferable. An example of such a system is presented at the conclusion of the paper.
